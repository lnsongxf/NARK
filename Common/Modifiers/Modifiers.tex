\documentclass[12pt]{econtex}
\usepackage{econtexSetup}\usepackage{econtexShortcuts}

\newcommand{\splitcell}[2][c]{%
	\begin{tabular}[c]{@{}c@{}}\strut#2\strut\end{tabular}%
}

\begin{document}

\begin{verbatimwrite}{./title.tex}
Recommendations for Variable Modifiers
\end{verbatimwrite}

\title{\input ./title}

\date{\today}
\maketitle 

\begin{verbatimwrite}{./body.tex}

The following are useful across many contexts:
\begin{table}[h]
	\centering
	\begin{tabular}{||>{\ttfamily}ccc||} 		
		\hline
   \textit{[object]}\texttt{Agg} & - & Value of something at the aggregate level (as opposed to \texttt{Ind})
\\ \textit{[object]}\texttt{Ind} & - & Value of something at the level of an individual (as opposed to \texttt{Agg})
\\ \textit{[object]}\texttt{Lvl} & - & Level 
\\ \textit{[object]}\texttt{Rto} & - & Ratio 
\\ \textit{[object]}\texttt{Bot} & - & Lower value in some range 
\\ \textit{[object]}\texttt{Top} & - & Upper value in some range 
\\ \textit{[object]}\texttt{Min} & - & Minimum possible value 
\\ \textit{[object]}\texttt{Max} & - & Maximum possible value 
\\ \textit{[object]}\texttt{Cnt} & - & Continuous-time value
\\ \textit{[object]}\texttt{Dsc} & - & Discrete-time value
\\ \textit{[object]}\texttt{Shk} & - & Shock 
\\ \textit{[object]}\texttt{Trg} & - & The `target' value of a variable 
\\ \textit{[object]}\texttt{Rte} & - & A `rate' variable like the discount rate $\discRte$
\\ \textit{[object]}\texttt{Fac} & - & A factor variable like the discount factor $\DiscFac$
\\ \textit{[object]}\texttt{Amt} & - & An amount, like \texttt{TaxAmt} which might be lump-sum
\\ 	\hline
	\end{tabular}
	\caption{General Purpose Modifiers}
	\label{table:General}
\end{table}	

Shocks will generally be represented by finite vectors of outcomes and their probabilities.  For example, permanent income is called \texttt{Perm} and shocks are designated \texttt{PermShk}
\begin{table}[h]
	\centering
	\begin{tabular}{||>{\ttfamily}ccc||} 		
		\hline
   \textit{[object]}\texttt{Prbs} & - & Probabilities of outcomes (e.g. \texttt{PermShkPrbs} for permanent shocks) 
\\ \textit{[object]}\texttt{Vals} & - & Values (e.g., for mean one shock \texttt{PermShkVals} . \texttt{PermShkPrbs} = 1) 
\\ 	\hline
	\end{tabular}
	\caption{Probabilities}
	\label{table:Probabilities}
\end{table}	



Timing can be confusing because there can be multiple ordered steps 
within a `period.'  We will use \texttt{Prev}, \texttt{Curr}, \texttt{Next} to refer
to steps relative to the local moment within a period, and $t$ variables to refer to succeeding periods:
\begin{table}[h]
	\centering 
	\begin{tabular}{||>{\ttfamily}ccc||} 		
		\hline
   \textit{[object]}tm2 & - & object in period $t$ minus 2 
\\ \textit{[object]}tm1 & - & object in period $t$ minus 1 
\\ \textit{[object]}Now & - & object in period $t$
\\ \textit{[object]}t   & - & object in period $t$ (alternative definition)
\\ \textit{[object]}tp1 & - & object in $t$ plus 1 
\\ \textit{[object]}tpn & - & object in $t$ plus $n$ 
\\ \textit{[object]}Prev & - & object in previous subperiod
\\ \textit{[object]}Curr & - & object in current subperiod
\\ \textit{[object]}Next & - & object in next subperiod
\\	\hline
	\end{tabular}
	\caption{Timing}
	\label{table:Timing}
\end{table}	


\end{verbatimwrite}
\input ./body.tex

\end{document}

For testing and debugging purposes, it is useful to compare numerical 
values constructed by the code to analytical results available in some
special cases.  To distinguish the corresponding object in the two cases,
we use
\begin{table}[h]
	\centering
	\begin{tabular}{||>{\ttfamily}ccc||} 		
		\hline
Anl & - & The analytical result
\\ Num & - & The numerical result 
\\ \hline
	\end{tabular}
	\caption{}
	\label{table:AnlVsNum}
\end{table}



  When measuring change over time, lower-case variables reflect rates
  while the corresponding upper-case variable is the corresponding
  factor connecting adjacent discrete periods.  (This
  convention rarely conflicts with the usage we endorse elsewhere of
  indicating individual-level variables by the lower and aggregate
  variables by the upper case).\footnote{If there is a need for the continuous-time
representation, we endorse use of the discrete-time rate defined below.  Any
author who needs a continuous-time rate, a discrete-time rate, and a discrete-time factor
is invited to invent their own notation.}

So, for example, if the annual interest rate is $\rfree=0.03$ or three percent, then
the annual interest factor is $\Rfree=1.03$.%\footnote{In the rare cases where it is necessary to distinguish between a continuous-time rate and a discrete-time rate -- for example, when there is an analytical result available in continuous time -- the variable in question can be modified by \texttt{Cnt} or \texttt{Dsc}.}

We predefine the following factors:
\begin{table}[h]
	\centering
	\begin{tabular}{|ccc|}
		\hline
Code    & Output & Description
\\ \hline
   \verb|\Rfree| & $\Rfree$     & Riskfree interest factor
\\ \verb|\rfree| & $\rfree$     & Riskfree interest return
\\ \verb|\Risky| & $\Risky$     & The return factor on a risky asset
\\ \verb|\risky| & $\risky$     & The return rate on a risky asset
\\ \verb|\Rport| & $\Rport$     & The return factor on the entire portfolio
\\ \verb|\rport| & $\rport$     & The return rate on the entire portfolio
\\	\hline
	\end{tabular}
	\caption{Factors}
	\label{table:Factors}
\end{table}

We depart from the upper-lower case scheme when the conventional
when the natural letter to use has an even more natural or urgent use elsewhere in our scheme.
A particularly common example occurs in the case of models like \cite{blanchardFinite} in which
individual agents are subject to a Poisson probability of death.  Because death was common in the
middle ages, we use the archaic Gothic font for the death rate; and the probability of survival is the cancellation of the probability of death:
\begin{table}[h]
	\centering
	\begin{tabular}{|>{\ttfamily}cccl|}
		\hline
		 Code & LaTeX & Description &  \\
		\hline
   \verb|\DieFac|     & $\DieFac$     & Proportion who die  &
\\ \verb|\LivFac|     & $\LivFac$     & Proportion who do not die $=(1-\DieFac)$ &
\\	\hline
	\end{tabular}
	\caption{Special Cases: Factors and Rates}
	\label{table:SpecialFactors}
\end{table}



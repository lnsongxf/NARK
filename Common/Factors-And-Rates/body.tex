
To make the mapping between continuous time and discrete time straightforward, our
convention is that lower-case variables reflect rates while the corresponding upper-case
variable is the corresponding factor over a discrete interval of time.

So, for example, if the annual interest rate is $\rfree=0.03$ or three percent, then
the annual interest factor is $\Rfree=1.03$.

We predefine the following factors:
\begin{table}[h]
	\centering
	\begin{tabular}{||>{\ttfamily}cccc||}
		\hline
		 Python Code & LaTeX Code & LaTeX Output & Description
\\ \hline  \texttt{Rfree}    & \verb|\Rfree|    & $\Rfree$    & The riskfree interest rate
\\ \texttt{Risky}    & \verb|\Risky|    & $\Risky$    & The return on a risky asset
\\ \texttt{Rport}    & \verb|\Rport|    & $\Rport$    & The return on the entire portfolio
\\	\hline
	\end{tabular}
	\caption{Factors}
	\label{table:Factors}
\end{table}

There are a few cases in which we must depart from the scheme in which lower case letters are
the rate associated with the corresponding upper case letter, most notably when the conventional
object is designated by a Greek letter that does not have a widely recognized lower case version.

\newcommand{\DiscFac}{\beta}
\newcommand{\discRte}{\vartheta}
\newcommand{\deprRte}{\delta}
\newcommand{\DieFac}{\pDead}
\newcommand{\LivFac}{\Alive}
\newcommand{\PopFac}{\PopGro}
\newcommand{\popRte}{\popGro}
\begin{table}[h]
	\centering
	\begin{tabular}{||>{\ttfamily}cccc||}
		\hline
		 Python Code & LaTeX Code & LaTeX Output & Description \\
		\hline
   \texttt{DeprFac} & \verb|\DeprFac|    & $\DeprFac$    & Depreciation factor
\\ \texttt{deprRte} & \verb|\deprRte|    & $\deprRte$    & Depreciation rate
\\ \texttt{DieFac}  & \verb|\DieFac|     & $\DieFac$     & Proportion who die
\\ \texttt{LivFac}  & \verb|\LivFac|     & $\LivFac$     & Proportion who do not die $=(1-\DieFac)$
\\ \texttt{DiscFac} & \verb|\DiscFac|    & $\DiscFac$    & The discount factor: $1/(1+\discRte)$
\\ \texttt{discRte} & \verb|\discRte|    & $\discRte$    & The discount rate: $\DiscFac^{-1}-1$
\\ \texttt{PopFac}  & \verb|\PopGro|     & $\PopGro$     & The growth factor for population
\\ \texttt{popRte}  & \verb|\popRte|     & $\popRte$     & The growth rate for population
\\	\hline
	\end{tabular}
	\caption{Special Cases: Factors and Rates}
	\label{table:SpecialFactors}
\end{table}



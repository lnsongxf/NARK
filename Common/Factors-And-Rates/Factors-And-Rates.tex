\documentclass{econtex}
\usepackage{econtexSetup}\usepackage{econtexShortcuts}

\begin{document}

\begin{verbatimwrite}{./title.tex}
Time Factors and Rates
\end{verbatimwrite}

\title{\input ./title}

\date{\today}
\maketitle 

\begin{verbatimwrite}{./body.tex}

  When measuring change over time, lower-case variables reflect rates
  while the corresponding upper-case variable connects adjacent discrete periods.\footnote{This
  convention rarely conflicts with the usage we endorse elsewhere of
  indicating individual-level variables by the lower and aggregate
  variables by the upper case.}$^{,}$\footnote{If there is a need for the continuous-time 
representation, we endorse use of the discrete-time rate defined below.  Any
author who needs a continuous-time rate, a discrete-time rate, and a discrete-time factor
is invited to invent their own notation.}  So, for example, if the annual interest rate is $\rfree=0.03$ or three percent, then 
the annual interest factor is $\Rfree=1.03$.%\footnote{In the rare cases where it is necessary to distinguish between a continuous-time rate and a discrete-time rate -- for example, when there is an analytical result available in continuous time -- the variable in question can be modified by \texttt{Cnt} or \texttt{Dsc}.}


\begin{table}[h]
	\centering
	\begin{tabular}{|ccc|} 		
		\hline
Code    & Output & Description 
\\ \hline 
   \verb|\Rfree| & $\Rfree$     & Riskfree interest factor
\\ \verb|\rfree| & $\rfree$     & Riskfree interest return
\\ \verb|\Risky| & $\Risky$     & The return factor on a risky asset
\\ \verb|\risky| & $\risky$     & The return rate on a risky asset
\\ \verb|\Rport| & $\Rport$     & The return factor on the entire portfolio
\\ \verb|\rport| & $\rport$     & The return rate on the entire portfolio
\\ \verb|\rport| & $\rport$     & The return rate on the entire portfolio
\\ \verb|\RSave| & $\RSave$     & Return factor earned on positive end-of-period assets
\\ \verb|\rsave| & $\rsave$     & Return rate earned on positive end-of-period assets
\\ \verb|\RBoro| & $\RBoro$     & Return factor paid on debts
\\ \verb|\rboro| & $\rboro$     & Return rate paid on debts 
\\	\hline
	\end{tabular}
	\caption{Factors and Rates}
	\label{table:Factors}
\end{table}	

We depart from the upper-lower case scheme when the natural letter to use has an even more natural or urgent use elsewhere in our scheme.
A particularly common example occurs in the case of models like \cite{blanchardFinite} in which 
individual agents are subject to a Poisson probability of death.  Because death was common in the 
middle ages, we use the archaic Gothic font for the death rate; and the probability of survival is the cancellation of the probability of death:
\begin{table}[h]
	\centering
	\begin{tabular}{|>{\ttfamily}cccl|} 		
		\hline
		 Code & LaTeX & Description &  \\ 
		\hline
   \verb|\DieFac|     & $\DieFac$     & Probabilty of death & 
\\ \verb|\LivFac|     & $\LivFac$     & Probability to not die $=(1-\DieFac)$ & 
\\	\hline
	\end{tabular}
	\caption{Special Cases: Factors and Rates}
	\label{table:SpecialFactors}
\end{table}	


\end{verbatimwrite}
\input ./body.tex


\input econtexBibMake

\end{document}
